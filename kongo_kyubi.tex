\section{金剛九尾 \- 〜 \- Kongou Kyuubi}

\subsection{貘 \- (Baku)}
\begin{center}
\begin{obeylines}
鬼(おに)の哭(な)く音(ね)が 繚乱(りょうらん)と舞(ま)い
煌(きら)めく刹那(せつな) 鳳凰(とり)が羽搏(はばた)く
夢幻(むげん)の裡(うち)に 點睛(てんせい)を遂(と)げ
魔天(まてん)の主(ぬし)が 物(もの)の怪(け)と成(な)る
\hfill

流例(るれい)の廉(かど)は 不退(ふたい)
無点(むてん)の春(はる)は 操(みさお)
溟海(めいかい)の果(は)て 遥(はるか) 跡絶(とだ)える 由無(よしな)く
歩(あゆ)ぶ 軈(やが)て 其処(そこ)に生(お)い立(た)つ
\hfill

刻(とき)の 愛(かな)しさに
此(こ)の眥(まなさき) 開(ひら)いて 往(ゆ)く丈(だけ)
岨(そわ)の 花笑(はなえ)みに 此(こ)の羅袖(らしゅう)が
染(そ)まりて 咲(さ)く哉(かな)
\hfill

底滓(そこり)は 目(め)ら向(む)けず 除(のぞ)き
祝詞(のりと)は 苦(にが)し甘露(かんろ)に 似(に)て
凶夢(まがゆめ) 斑(むら)し 貘(ばく)が 餌(え)ばみ
幻(まぼろし) 瞬(またた)く 刹那(せつな)に 鏘鏘(そうそう)
鳳凰(とり)は飛(と)び立(た)つ
\hfill

刻(とき)の 愛(かな)しさに
此(こ)の眥(まなさき) 開(ひら)いて 往(ゆ)く丈(だけ)
岨(そわ)の 花笑(はなえ)みに 此(こ)の和酬(わしゅう)で
応(こた)え続(つづ)けよう
\hfill

疾(と)く 醒(さ)めて 見上(みあ)げれば 抉(くじ)られたら 天(そら)
賓(まれうど)の 土産(みあげ)も 解(と)かざる儘(まま)で
満開(まんかい)の葉花(はばな)が 絶界(ぜっかい)を照(て)らすとき
光(ひかり)が 溢(あふ)れる
刻(とき)の 愛(かな)しさに 此(こ)の眥(まなさき)
開(ひら)いて 往(ゆ)く丈(だけ)
岨(そね)の 花笑(はなえ)みに 此(こ)の羅袖(らしゅう)が
染(そ)まりて 咲(さ)く哉(かな)
\hfill

鳳凰(とり)よ 青竜(せいりゅう)よ
其(そ)の翼(つばさ)は 辞(いな)びて 歩(あゆ)もう
路(みち)の あらましを 此(こ)の声(こえ)で 歌(うた)い伝(つた)えよう
\end{obeylines}
\end{center}

%========================================

\subsection{蒼き独眼 \- (Aoki Dokugan)}
\begin{center}
\begin{obeylines}
戯(ざ)れに 縺(もつ)れし 糸に 箍(たが) 抄(すく)い 取られ
疾(と)うに 無くした 神の綱 只 己を 懸けて
抗(あらが)うは 穢(けが)れ無き 守(かみ)の 真名(まな)
遺(のこす)す為 誓いの 縁(えにし)で 遂(と)ぐ
降(くだ)るを 悔(く)ゆる 惑い 其(そ)は 武人の 性(さが)
理(ことわり)も無き 為置(しおき)なら もう 情(こころ)は 要らぬ
抗(あらが)うは 細(さざれ)なる 波の 未(ま)だ 果てぬ為
終焉(おわり)を 背にして 立つ
差し零(あや)す 蹤血(はかり)の 鮮(あざ)やぐ 紅(くれない)
哉(かな) 運命(さだめ)の 証(あかし)
何も彼(か)もが 呑まれゆく 望(のぞ)まざる 紮(から)げりに
只(ただ)術(すべ)も無く 覆(おお)うは 大牙(たいが)の 闇
叫びも 掠(かす)れど曇り無き 此の 蒼き 眼差しを 今
月に代(か)え 崎嶇(きく)など 斬り捨て 憚(はばか)る
\hfill

Zare ni motsureshi ito ni taga sukui torare
Touni nakushita kami no tsuna tada onore wo kakete
\hfill

Aragau wa kegare naki kami no mana
Nokosu tame chikai no enishi de togu
\hfill

Kudaru wo kuyuru madoi soha bujin no saga
Kotowari mo naki shioki nara mou kokoro wa iranu
\hfill

Aragau wa sazarenaru nami no mada hatenu tame
Owari wo senishite tatsu
\hfill

Sashiayasu hakari no azayagu kurenai
Kana sadame no akashi
Nanimo kamo ga
\hfill

Nomareyuku nozomazaru karageri ni tada
sube mo naku oou wa taiga no yami
Sakebi mo kasuredo
kumorinaki kono aoki manazashi wo ima
Tsuki ni kae
kiku nado kirisute habakaru
\end{obeylines}
\end{center}

%========================================

\subsection{相剋 \- (Soukoku)}
\begin{center}
\begin{obeylines}
ruten no hate nite okasareta no ha
kinki no mitsugetsu
otome no umekusa nitou no kagema
uma(f)u ha sue no ouka
\hfill

``uramu koto de nani wo hataseru no''
madou koe ha tada tooku
yoki koto kiku
sono negai nakaba de
kubi yo odore
\hfill

geten no rukeichi uzumoreta noha
sange no mokuyoku
otome no rakuin madoi no onigo
shuen ni oitatsu
\hfill

``ubau koto de nani wo erareru no''
saka(f)u koe ha tada hibiku
yoki koto kiku
kono te ni kaeru made
kubiri warau
\hfill

``uramu koto de nani wo hataseru no''
madou koe ha mada tooku
uki koto kike
kono kuroi kashiri yo
yami ni shizume
\end{obeylines}
\end{center}

%========================================