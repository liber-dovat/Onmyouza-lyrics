\section{煌神羅刹 \- 〜 \- Kojin Rasetsu}

\subsection{羅刹 \- (Rasetsu)}
\begin{center}
\begin{obeylines}
寧悪(ねいあく)なる貌形(ほうぎょう) 闇に紛れて
怨み辛(つら)み纏(まと)いて立つ
\hfill

静寂(せいじゃく)なる真秀(まほ)ら場(ば) 酸鼻(さんび)を極め
月の貌(かお)も朱に染まる
\hfill

暴(ぼう) 憎(そう) 念(ねん)
血達磨(ちだるま)の族(うから) 呪いを込めて
烙印(らくいん) 押さるる鬼は
\hfill

忘れじの追儺(ついな)と紊(みだ)る汚吏(おり)の流れ
聨亘(れんこう)の罪 連れを枕(ま)かれ
手くろもの相応(ふさい)の拷(ごう)を以(も)ち贖(あがな)え
淵謀(えんぼう)の荼毘(だび) 怨(おん)は絶えぬと
\hfill

啓白(けいはく)する内憤(ないふん) 神に疎(うと)まれ
継(つ)ぎの吾子(あこ)も呆気(あけ)に縊(くび)れる
\hfill

暴 憎 念
火達磨(ひだるま)族 救い求めて
経絡(けいらく) 突かるる餓鬼(がき)は
\hfill

忘れじの追儺と紊る汚吏の流れ
聨亘の罪 連れを枕かれ
手くろもの相応の拷を以ち贖え
淵謀の荼毘 怨は絶えぬと
\hfill

暴 憎 念
茹(う)だる魔の嬰児(えいじ) 挿(す)げる鬼殿(おにどの)
脈々 続く蛇道(じゃどう)は
\hfill

忘れじの追儺と紊る汚吏の流れ
聨亘の罪 連れを枕かれ
手くろもの相応の拷を以ち贖え
淵謀の荼毘 怨は絶えぬと
\hfill

Neiaku naru bougyou yami ni magirete
urami tsurami
matoitetatsu
\hfill

Seijaku naru mahoraba sanbi wo kiwame
tsuki no kao mo ake ni somaru
\hfill

Bou zou nen
chidarama no ukara noroi wo komete
rakuin osaruru oni
wa
\hfill

Wasureji no tsui na to midaru ori no nagare
renkou no tsumi tsure wo makare
te kuro mono fusai no gou
wo mochi aganae
enbou no tabi on wa taenu to
\hfill

Kei hakusuru naifun Kami ni utomare
tsugi no ako mo akke
ni kubireru
\hfill

Bou zou nen
hidaruma no ukara sukuimotomete
keiraku tsukaruru gaki wa
\hfill

Wasureji no
tsui na to midaru ori no nagare
renkou no tsumi tsure wo
makare
te kuro mono fusai no gou wo mocchi aganae
enbou no tabi on wa taenu to
\hfill

Bou zou nen
udaru
ma no eiji sugeru oni-dono
myakumyaku tsuzuku jadou wa
\hfill

Wasureji no tsui na to midaru ori no nagare
renkou
no tsumi tsure wo makare
te kuro mono fusai no gou wo
mochi aganae
enbou no tabi on wa taenu to
\end{obeylines}
\end{center}

%========================================

\subsection{煌 \- (Kirameki)}
\begin{center}
\begin{obeylines}
閉ざされた視界を 詛いでこじ開けて
隠された欺瞞を 白日に晒し上げ
此の胸の胎芽は 迸る炎と
今 煌を放つ魔魅へと
腫れ上がり輝き出す
込み上げた想いを 思う様吐き出して
振り上げた拳を(鉄槌)叩き付けろ
此の胸の胎芽は 迸る炎と
今 煌を放つ魔魅へと
腫れ上がり輝き出す
閉ざされた視界を 詛いでこじ開けて
隠された欺瞞を 白日に晒し上げ
心の裡で響く英霊の詞
其の胸の怪訝は 軈て来る淘げと
今 煌を放つ魔魅へと
此の胸の胎芽は 迸る炎と
今 煌を放つ魔魅へと
腫れ上がり輝き出す
\hfill

Tozasareta shikai wo
majinai degoji akete
kakusareta giman wo
hakujitsu ni sarashi age
\hfill

kono mune no taiga wa
hotobashiru homura to
ima kirameki wo hanatsu mami e to
hare agari kagayakidasu
\hfill

Komiageta omoi wo
omousama hakidashite
furiageta kobushi wo tettsui
tatakitsukero
\hfill

kono mune no taiga wa
hotobashiru homura to
ima kirameki wo hanatsu mami e to
hare agari kagayakidasu
\hfill

Tozasareta shikai wo
ajinai degoshi akete
kakusareta giman wo
hakujitsu ni sarashi age
\hfill

kokoro no uchi de
hibiku eirei no kotoba
\hfill

sono mune no taiga wa
yagate kuru yonage to
ima kirameki wo hanatsu mami e to
kono mune no taiga wa
hotobashiru homura to
ima kirameki wo hanatsu mami e to
hare agari kagayakidasu
\end{obeylines}
\end{center}

%========================================

\subsection{月に叢雲花に風 \- (Tsuki ni Murakumo Hana ni Kaze)}
\begin{center}
\begin{obeylines}
謦が囁いている
Koe ga sasayaiteiru
翳りの淵は身悶える
Kageri no fuchi wa mimodaeru
時が轟いている
Toki ga todoroiteiru
路傍の人は行き過ぎる
Robou no hito wa yukisugiru
\hfill

啓示の月が夢に舞う
Keiji no tsuki ga yume ni mau
一瞬の刻を
Isshun no toki wo
花に生まれて甘に咲いて
Hana ni umarete uma ni saite
慶事の月が雨に啼く
Keiji no tsuki ga ame ni naku
一瞬の刻を
Isshun no toki wo
雲に焦がれて風は凪いで
Kumo ni kogarete kaze wa naide
\hfill

末那が揺らめいている
Mana ga yurameiteiru
滾りの韃は翻る
Tagiri no muchi wa hirogaeru
澱が蠢いている
Ori ga ugomeiteiru
返しの前に短くなる
Kaeshi no maeni mizokunaru
\hfill

啓示の月が夢に舞う
Keiji no tsuki ga yume ni mau
一瞬の刻を
Isshun no toki wo
花に生まれて甘に咲いて
Hana ni umarete uma ni saite
慶事の月が雨に啼く
Keiji no tsuki ga ame ni naku
一瞬の刻を
Isshun no toki wo
雲に焦がれて風は凪いで
Kumo ni kogarete kaze wa naide
\hfill

華やいだ虚飾の風の宿りから
Hanayaida kyoshoku no kaze no yadori kara
雲の切れ間仰ぎ謳う
Kumo no kirema aogi utau
\hfill

啓示の月が夢に舞う
Keiji no tsuki ga yume ni mau
一瞬の刻を
Isshun no toki wo
花に生まれて甘に咲いて
Hana ni umarete uma ni saite
慶事の月が雨に啼く
Keiji no tsuki ga ame ni naku
一瞬の刻を
Isshun no toki wo
雲に焦がれて風は凪いで
Kumo ni kogarete kaze wa naide
\end{obeylines}
\end{center}

%========================================

\subsection{組曲「黒塚」〜安達ヶ原 \- (Kumikyoku ``Kurozuka'' 〜 Adachi-ga-hara)}
\begin{center}
\begin{obeylines}
Kaze no to no tooki inishie no toga yo
Ima wa musubo horu mizuchi no tama ka
Ko no kure no yami ni hisomite nagaraeba
Koko nagara yomotsu goku tonari nuru
\hfill

Kasokeshi hito no kokoroba e
Majiro oni no sasameki
\hfill

Kuchi nokoru hone wa nanzo shiroki iro ya
Nozarashi tonarite na ho usuwarau
\hfill

Kasokeshi hito no kokoroba e
Majiro oni no sasameki
Soko hinaki tokoyami ni otsu
Chizomaru kinu wo matoite
Ware wa samo kimi tonari keri
Chi wo susuri niku wo kurafu
Ware wa samo kimi tonari keri chizoba e te tsuma wo kakagu
\hfill

Are hodo mite wa naranu to moushita ni toutou
Kono sugata wo mirarete shimouta
Anata-sama mo kono baba to kakazurouta no ka
Un no tsuki no aki to akiramena saru ga ee
Ikanimo ruirui to tsumoru shiro hone wa
Watashi no kuirouta hito no nareno hate
Watashi mo mukashi wa kako utsukushuu gozaimashita
Seoi kirenu hodo no gouzai ga watashi wo oni ni shita no de gozaimasu
Itsushika kuchi wa sake kao wo minikuku yugami kami wa mizu borashii hakuhatsu
Ni nari hateshimashita
Hitoyoru no yado wo to tazune kita tabibito no nodobue
Ni tsume wo tate chi wo susuri
Sono niku wo kuiroute kyou made ikinagara e ta no de gozaimasu
Naze konoyouna gou wo seouta ka sore wa watashi ga shinda wagako wo
Kuroute shimou takara desu
\end{obeylines}
\end{center}

%========================================

\subsection{組曲「黒塚」〜鬼哭啾々 \- (Kumikyoku ``Kurozuka'' 〜 Kikokushuushuu)} % \- (Kumikyoku \- ``Kurozuka'' \- ~ Kikokushuushuu)
\begin{center}
\begin{obeylines}
Chishio taru yaiba no gotoki
kokoro wa shin i ni fusubite
\hfill

Awarenaru wa gou ni tada umeku sumeku oni yo
waga te de yomiji he to michibiku
\hfill

Towa ni semegu sadame ni naite kono tsutsu yami wo
nagaru saki ni hikari wa mienu
\hfill

Tori no ne wa tooku fukishiku kaze ni kiyu
sugishi hi no sugata horohoro to chirinuru
\hfill

Wanana kedo hitori no zura ni tatazumi
fumi madou hana ni natsukashiki uta wa hibiku
\hfill

Sarakedasu nageki wo tada uchi furuwasu oni yo
sono te wo haraite kiyomen to
\hfill

Towa ni taburu sadame wo saite gouma ni otsuru
sore wa kuraki yami wo chigiru
\hfill

Yasurakeshi hikari wo shirite toburai wa gou wo tokisaku
sono me ni namida wa afururu
\end{obeylines}
\end{center}
