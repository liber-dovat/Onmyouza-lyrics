\section{鬼哭転生 \- 〜 \- Kikoku Tenshou}

\subsection{氷の楔 \- (Kouri no Kusabi)}
\begin{center}
\begin{obeylines}
いつか結ぶ運命の糸は
Itsuka musubu sadame no ito wa
遠く彼岸の旅路と成りて…
tooku higan no tabiji to narite...
\hfill

愛しき人の骸を横たえて
Itoshiki hito no mukuro wo yokotaete
血も通わぬ此の腕を齧る
chi mo kayowanu kono ude wo kajiru
降りゆく雪の白さに怯えては
furiyuku yuki no shirosa ni obiete wa
終の知らせを待ちわびる
tsui no shirase wo machiwabiru
\hfill

己が罪の深さを知りて
Ono ga tsumi no fukasa wo shirite
君を殺めた指を落とすとも
kimi wo ayameta yubi wo otosu tomo
痛みも感じぬ氷の身は
itami mo kanjinu koori no mi wa
命を絶つ事も許されず
inochi wo tatsu koto mo yurusarezu
雪の化身と生まれし業を
yuki no keshin to umareshi gou wo
背負いて永遠に哭き続ける
seoite towa ni naki tsuzukeru
\hfill

愛する者達を抱き締める事さえ
Aisuru mono-tachi wo dakishimeru koto sae
叶わぬ孤独を生きるなら
kanawanu kodoku wo ikirunara
紅く燃えさかる業火で此の身を
akaku moesakaru gouka de kono mi wo
焼かれて地獄へと堕ちたい
yakarete jigoku e to ochitai
\hfill

雪は何処までも 白く降り積もる
Yuki wa dokomademo shiroku furitsumoru
それは 終りなき罰の様に
sore wa owarinaki batsu no you ni
\hfill

愛する者達を抱き締める事さえ
Aisuru mono-tachi wo dakishimeru koto sae
叶わぬ孤独を生きるなら
kanawanu kodoku wo ikiru nara
どうか燃えさかる業火で此の身を
douka moesakaru gouka de kono mi wo
焼き尽くし灰にして欲しい
yakitsukushi hai ni shite hoshii
\hfill

いつか結ぶ運命の糸は
Itsuka musubu sadame no ito wa
固く氷の楔と成りて…
kataku koori no kusabi to narite...
\end{obeylines}
\end{center}

%========================================

\subsection{鬼斬忍法帖 \- (Onikiri Ninpouchou)}
\begin{center}
\begin{obeylines}
粉雪の舞い踊る
Kona yuki no maiodoru
寒の殿戸の下
kan no tonoto no shita
匂やかな妖気立つ
nioyakana youkitatsu
陰に潜みし影
kage ni hisomishi kage
\hfill

魔の物に魅入られし
Ma no mono ni miirareshi
人の形の鬼
hito no katachi no oni
殺陣は血で煙り
satsujin wa chi de kemuri
屍は山と成る
kabane wa yama to naru
\hfill

魂亡くした 虚ろな器
Kokoro nakushita utsuro na utsuwa
玉虫色の 幻に包みて
tamamushiiro no maboroshi ni tsutsumite
\hfill

鬼斬忍法
Onikiri ninpou
\hfill

咲いた側散りぬるは
Saita soba chiri nuru wa
邪気を孕みし花
jaki wo haramishi hana
嫋やかな魔性の力
taoyaka na mashou no chikara
病みを飲み込む闇
yami wo nomikomu yami
\hfill

魂亡くした 虚ろな器
Kokoro nakushita utsuro na utsuwa
玉虫色の 幻に包みて
tamamushiiro no maboroshi ni tsutsumite
\hfill

鬼斬忍法
Onikiri ninpou
\hfill

二つに裂いても 微塵に刻めど
Futatsu ni saite mo mijin ni kizamedo
内から外から 鬼は潜み入る
uchi kara soto kara oni wa hisomi iru
人の弱さ故 懐柔さるるが
hito no yowasa yue kaijuu saruruga
己が魂で 打ち砕け鬼を
ono ga tamashii de uchikudake oni wo
\end{obeylines}
\end{center}

%========================================

\subsection{百の鬼が夜を行く \- (Hyaku no Oni ga Yoru wo Yuku)}
\begin{center}
\begin{obeylines}

(\ruby{百鬼}{ひゃっき})夜闇を切り裂いて
\ruby{有象無象}{うぞうむぞう}の\ruby{異形}{いぎょう}が
(百鬼)\ruby{練}{ね}り歩く月一度の
我が物顔の\ruby{鹵簿}{ろぼう}
\hfill

\ruby{触}{は}え\ruby{尽}{つ}く京の都に
哀れに横たわる\ruby{骸}{むくろ}
この世に残した怨み
\ruby{幾許}{いくばく}か晴らさんと
\hfill

月が燃え尽きた天の火が 赤と黒の下
溢れ出した\ruby{百}{もも}の鬼が 我先と夜を行く
\hfill

(百鬼)\ruby{天変地異}{てんぺんちい}の前触れ
\ruby{己}{おの}が\ruby{所業}{しょぎょう}の代償
(百鬼)逃げ出す事も叶わず
あれよあれよの\ruby{頓死}{とんし}
\hfill


\ruby{陰陽}{いんよう}の狭間から
響き渡る笑い声
あの世で結んだ\ruby{契}{ちぎ}り
\ruby{永久}{とこしえ}に忘れじと
\hfill

月が燃え尽きた天の火が 赤と黒の下
溢れ出した百の鬼が \ruby{挙}{こぞ}り夜を行く
闇が踊りだす\ruby{巳}{み}の日の \ruby{弥生}{やよい}の空には
溢れ出した百の鬼が 我先と夜を行く
\hfill


\ruby{矮小}{わいしょう}なり、\ruby{姑息}{こそく}なり
憎き藤原、\ruby{醍醐}{だいご}の一族
既に亡き者\ruby{時平}{ときひら}に
代わりて\ruby{屠}{ほふ}る\ruby{子々孫々}{ししそんそん}ども
\end{obeylines}
\end{center}

%========================================

\subsection{陰陽師 \- (Onmyouji)}
\begin{center}
\begin{obeylines}
``Toukai no kami, na wa amei
seikai no kami, na wa shukuryou
nankai no kami, na wa kyojou
hokkai no kami, na wa gukyou
shikai no taishin, zenchi zennou no chikara o motte
hyakki wo shirizoke, kyouzai wo harau
ware, tsuneni gesshou wo motte senji ni kuwa e
nisshin inyou wo miru mono nari''
\hfill

Araburu shinra no mamono-tachi yo
ra go no yoru ni mezamen
kakageru kikyou no hoshi no ue ni
hirakaru hikari to kage no mon
\hfill

Yami yori idetaru mashou wo harai saru
hikari hanatsu juuni shinshou
waga ashi ni tsudoi noroi wo komete tobe
kono yo no wa wo musuban ga tame
\hfill

Hikari wa yami wo saki kaze wa kumo wo chirasu
banshou no chikara yo kono ryoute no chikara to kaware
\hfill

Uzumaku kurakumo ten wo koroshi
arawaru kyouji no unarika
jigoku ni haizuru mushi no gotoku
notautsu aware naru onryou
\hfill

Maiodoru jakini shu no reifu wo hanachi
meido okuri no uta wo utau
wananaku mamira wo kurau shikigami domo
hone wo hami chi niku wo susure yo
\hfill

Hikari wa yami wo saki kaze wa kumo wo chirasu
banshou no chikara yo kono ryoute no chikara to kaware
\hfill

``Ima wa mukashi kyou no miyako ni
hito naranu chikara wo ayatsuru mono ari
Furuido yori meikai e to ikidashi,
shisha to kata mono mokke to tawamuru
amata no shikigami wo shiekishi,
mangetsu no yoru ni wa mamono ni matagari
ten wo kaketa to iu
sono mono wo hito wa onmyouji to yobu''
\hfill

Yami yo ni ayanasu chimimouryou no mure
meifu no mukuzu to hikisakan
todoroku raimei sono sakebi no hate ni
guren no hi wo ageru jigoku e
\hfill

Kaere ma no tami yo nidoto mezamenu you
musunda in ni nasake wo kome
shizumaru tamashii yomi no soko de nemure
awarenaru saga wo wasuresari
\hfill

Hikari wa yami wo saki kaze wa kumo wo chirasu
banshou no chikara yo kono ryoute no chikara to kaware
\end{obeylines}
\end{center}

%========================================

\subsection{亥の子唄 \- (I no Kouta)}
\begin{center}
\begin{obeylines}
Koko mo hitotsu ni iwaimasho
\hfill

Ichidetara funmai te
Ni de nikkori waru ote
Sande saketsutte
Yottsu yo no naka yoi yoi ni
\hfill

Ittsu itsumon gootooku ni
Muttsu mubyou sokusai ni
Natsu nani gotonai you ni
Yatsu yashiki wo tate narabe
Koko no tsukokura wo tate hiroge
Toode toutou osamatta
Hon hon eei
\hfill

Medetai na medetai na
Medetai mono wa osensu yo
Osensu kaname ni ikehorite
Ike no shita ni taoshitsuke
Sono ta ni taoshite karu toki nya
Hitokuro kareba nisengoku
Futakuro kareba shisengoku
Mikuro mo kareba koku shiranu
Sono komesake ni tsukushite
Sake wa jouzake izumi sake
Sono sake ippai non damon nya
Mannochou ja tonari sou na
Hon hon eei
\hfill

Koko no yashiki wa yoi yashiki
Koko no kodomo wa yoi kodomo
\end{obeylines}
\end{center}
